\documentclass[a4paper,14pt]{extarticle}
\usepackage[left = 20mm, right = 20mm, top = 25mm, bottom = 25mm]{geometry}
\usepackage{amssymb}
\usepackage{enumerate}
\usepackage{latexsym}
\usepackage{amsmath}
\usepackage{euscript}
\usepackage{graphics}
\usepackage[T1, T2A]{fontenc}
\usepackage[utf8]{inputenc}
\usepackage[russian]{babel}

\usepackage[usenames]{color}
\usepackage{colortbl}
\usepackage{pgf,tikz}
\usepackage{multicol}
\sloppy

\usepackage{hyperref} %сделать ссылки интерактивными
\hypersetup{
colorlinks = true, %цветные ссылки вместо рамок
urlcolor = blue, %внешние ссылки
linkcolor = blue, %внутренние ссылки
citecolor = red %ссылки на литературу
}

\newcommand{\eqdef}{\overset{\mathrm{def}}{=\joinrel=}}
\def\re{{\rm Re}}
\def\im{{\rm Im}}
\def\dim{\rm dim}
\def\Ext{\rm Ext}
\def\wt#1{{{\widetilde #1} }}
\def\wh#1{{{\,\widehat #1\,} }}
\def\graph{{\rm gr\,}}
\def\ran{{\rm ran\,}}
\def\dom{{\rm dom\,}}
\def\ker{{\rm ker\,}}
\def\supp{{\rm supp\,}}
\def\diag{{\rm diag\,}}

\newcommand\dN{{\mathbb{N}}}
\newcommand\dR{{\mathbb{R}}}
\newcommand\dC{{\mathbb{C}}}
\newcommand{\bO}{{\mathbb{O}}}
\newcommand{\bU}{{\mathbb{U}}}
\newcommand\dZ{{\mathbb{Z}}}

\newcommand\gotB{{\mathfrak{B}}}
\newcommand\gotD{{\mathfrak{D}}}
\newcommand\gotH{{\mathfrak{H}}}
\newcommand\gotK{{\mathfrak{K}}}
\newcommand\gotL{{\mathfrak{L}}}
\newcommand\gotM{{\mathfrak{M}}}
\newcommand\gotN{{\mathfrak{N}}}
\newcommand\gotR{{\mathfrak{R}}}
\newcommand\gotS{{\mathfrak{S}}}
\newcommand\gotT{{\mathfrak{T}}}
\newcommand\gott{{\mathfrak{t}}}
\newcommand\gotC{{\mathfrak{C}}}
\newcommand\gotZ{{\mathfrak{Z}}}
\newcommand{\RNumb}[1]{\uppercase\expandafter{\romannumeral #1\relax}}

\newcommand{\ga}{{\alpha}}
\newcommand{\gd}{{\delta}}
\newcommand{\gD}{{\Delta}}
\newcommand{\gga}{{\gamma}}
\newcommand{\gG}{{\Gamma}}
\newcommand{\gF}{{\Phi}}
\newcommand{\gf}{{\phi}}
\newcommand{\gk}{{\kappa}}
\newcommand{\gK}{{\Kappa}}
\newcommand{\gl}{{\lambda}}
\newcommand{\gL}{{\Lambda}}
\newcommand{\gO}{{\Omega}}
\newcommand{\go}{{\omega}}
\newcommand{\gs}{{\sigma}}
\newcommand\gS{{\Sigma}}
\newcommand{\gT}{{\Theta}}
\newcommand{\gY}{{\Upsilon}}

\newcommand\cA{{\mathcal{A}}}
\newcommand\cB{{\mathcal{B}}}
\newcommand\cC{{\mathcal{C}}}
\newcommand\cD{{\mathcal{D}}}
\newcommand\cH{{\mathcal{H}}}
\newcommand\cK{{\mathcal{K}}}
\newcommand\cM{{\mathcal{M}}}
\newcommand\cN{{\mathcal{N}}}
\newcommand\cO{{\mathcal{O}}}
\newcommand\cP{{\mathcal{P}}}
\newcommand\cT{{\mathcal{T}}}
\newcommand\cU{{\mathcal{U}}}
\newcommand\cZ{{\mathcal{Z}}}


\newcommand\res{{\rm res}}

\DeclareMathOperator{\sign}{sign}

\newtheorem{theorem}{Теорема}[subsection]
\newtheorem{lemma}{Лемма}[subsection]
\newtheorem{corollary}[theorem]{Следствие}
\newtheorem{definition}{Определение}[subsection]
\newtheorem{example}{Пример}[subsection]
\newtheorem{remark}{Замечание}[subsection]

\newcommand{\ba}{\begin{array}}
\newcommand{\ea}{\end{array}}

\newcommand{\bea}{\begin{eqnarray}}
\newcommand{\eea}{\end{eqnarray}}

\newcommand{\bead}{\begin{eqnarray*}}
\newcommand{\eead}{\end{eqnarray*}}

\newcommand{\be}{\begin{equation}}
\newcommand{\ee}{\end{equation}}

\newcommand{\bed}{\begin{displaymath}}
\newcommand{\eed}{\end{displaymath}}

\newcommand{\bl}{\begin{lemma}}
\newcommand{\el}{\end{lemma}}

\newcommand{\bt}{\begin{theorem}}
\newcommand{\et}{\end{theorem}}

\newcommand{\bc}{\begin{corollary}}
\newcommand{\ec}{\end{corollary}}

\newcommand{\br}{\begin{remark}}
\newcommand{\er}{\end{remark}}

\newcommand{\bd}{\begin{definition}}
\newcommand{\ed}{\end{definition}}

\newcommand{\bspi}{\begin{split}}
\newcommand{\espi}{\end{split}}

\newcommand{\la}{\label}
\newcommand{\rpm}{\raisebox{.2ex}{$\scriptstyle\pm$}}


\newenvironment{proof}%
{\begin{sloppypar}\noindent{\bf Доказательство.}}%
{\hspace*{\fill}$\square$\end{sloppypar}}
\renewcommand{\Large}{\fontsize{16}{25pt}\selectfont}

\newcommand{\slim}{\,\mbox{\rm s-}\hspace{-2pt} \lim}
\newcommand{\wlim}{\,\mbox{\rm w-}\hspace{-2pt} \lim}
\newcommand{\olim}{\,\mbox{\rm o-}\hspace{-2pt} \lim}
\newcommand{\transpose}[1]{\ensuremath{#1^{\scriptscriptstyle t}}}



\usepackage{indentfirst} 
\setlength\parindent{1cm}
% НАСТРОЙКИ ЗАГОЛОВКОВ
\usepackage{textcase} 
\usepackage{titlesec}
\titleformat{\section}[block]{\sffamily\Large\bfseries\filcenter}{\thesection}{0.5em}{}
\titleformat{\subsection}[block]{\large\sffamily\bfseries}{\thesubsection}{0.5em}{}
\titlespacing{\subsection}{2cm}{1mm}{3mm}
\begin{document}


\tableofcontents
\newpage

\section{Введение}
Современные задачи математического моделирования часто приводят к так называемым дифференциальным уравнением. Дифференциальное уравнение -- это уравнение, связывающее между собой неизвестную функцию, ее производные и аргумент. 

Простейшие дифференциальные уравнения известны нам еще из школы, например, с уроков физики. Так, если $S(t)$ -- путь, пройденный материальной точкой за время $t$ со скоростью $v(t)$, то 
\be\la{eq}
S'(t) = v(t),
\ee
и мы приходим к дифференциальному уравнению на функцию $S$. Если, скажем, функция $v(t)$ оказывается непрерывной при $t \geq 0$, то все решения уравнения (\ref{eq}) задаются как
$$
S(t) = \int\limits_0^t v(\tau) \, d\tau + C, \quad C \in \mathbb R.
$$
То, что мы получили целое семейство решений уравнения (\ref{eq}) не должно удивлять. Для нахождения конкретной функции нам нужно, например, знать начальное положение материальной точки или, что то же самое, значение $S(0)$. Положив $S(0)= 0$, мы приходим к так называемому частному решению рассматриваемого уравнения:
$$
S(t) = \int\limits_0^t v(\tau) \, d\tau.
$$
Исследование разрушений биологических клеток под действием ультразвука высокой интенсивности приводит к уравнению
$$
N'(t) = -kN(t),
$$
где $t$ -- время, $N$ -- концентрация живых клеток, а $k$ -- коэффициент, отвечающий за вероятность разрыва клетки в единицу времени. Нетрудно проверить, что функции
$$
N(t) = Ce^{-kt}, \quad C \in \mathbb R,
$$
удовлетворяют этому уравнению. Можно доказать, что других решений написанное уравнение не имеет. 

Приведенные выше уравнения содержат лишь первую производную неизвестной функции. Такое бывает не всегда. Например, уравнение, описывающее колебание шарика, подвешенного к потолку на пружине, отклоненного от положения равновесия, может быть записано так:
$$
x''(t) + \omega^2 x(t) = 0.
$$
Здесь $x(t)$ -- это отклонение шарика от положения равновесия в момент времени $t$. Можно доказать, что все решения заявленного уравнения могут быть записаны следующим образом:
$$
x(t) = C_1\cos (\omega t) + C_2 \sin (\omega t), \quad C_1, C_2 \in \mathbb R.
$$
Заметим, что для нахождения конкретного решения задать лишь $x(0)$ будет недостаточно. Нужно задать и, например, $x'(0)$ -- скорость в начальный момент времени. 

Понятно, что теория дифференциальных уравнений занимается не только и не столько нахождением решений этих самым уравнений, но и исследованием вопросов существования, единственности, аналитических свойств зависимости от параметров и многим другим. 

\section{Основные определения об уравнених первого порядка}
\subsection{Уравнение первого порядка и его решение. Уравнение в нормальной форме. Ломаные Эйлера}
Начнем с основных определений. 
\begin{definition}
Дифференциальным уравнением первого порядка называется уравнение вида
\be\la{de1}
F(x, y, y') = 0.
\ee
\end{definition}
Введем понятие решения дифференциального уравнения.
\begin{definition}
Решением дифференциального уравнения (\ref{de1}) на $\langle a, b \rangle$ называется произвольная функция $\varphi(x)$, что
$$
F(x, \varphi(x), \varphi'(x)) \equiv 0, \quad x \in \langle a, b \rangle.
$$
\end{definition}
Дадим и еще несколько часто встречающихся определений.
\begin{definition}
Множество всех решений дифференциального уравнения называется его общим решением.	
\end{definition}
\begin{definition}
Конкретное решение дифференциального уравнения называется его частным решением.	
\end{definition}
\begin{example}
Рассмотрим дифференциальное уравнение
$$
y' = \frac{1}{x^2}.
$$
Понятно, что у этого уравнения можно выделить семейства решений
$$
y(x) = -\frac{1}{x} + C, \quad C \in \mathbb R,
$$
заданные (максимально) либо на $(-\infty, 0)$, либо на $(0, +\infty)$. Найденное решение является общим.

Решения
$$
y = -\frac{1}{x}, \quad y = -\frac{1}{x} + \pi,
$$
заданные на соответствующих промежутках, являются частными
\end{example}
Решение дифференциального уравнения далеко не всегда может быть записано в явном виде.
\begin{definition}
Общим интегралом уравнения (\ref{de1}) называется соотношение вида
$$
\Phi(x, y, C) = 0,
$$
неявно задающее решение уравнения при некоторых значениях $C$.
\end{definition}
\begin{example}
Можно проверить, что общий интеграл уравнения 
$$
x + yy' = 0
$$
задается соотношением $x^2 + y^2 = C$, $C > 0$.
\end{example}
\begin{remark}
Общий интеграл уравнения вовсе не всегда описывает его, уравнения, общее решение. И вообще, определения общего решения, общего интеграла в разных источниках могут отличаться.
\end{remark}


\begin{definition}
График решения уравнения (\ref{de1}) называется интегральной кривой.	
\end{definition}
Нас часто будет интересовать так называемое дифференциальное уравнение в нормальной форме.
\begin{definition}
Уравнение вида
\be\la{de1n}
y' = f(x, y)
\ee
называется дифференциальным уравнением первого порядка в нормальной форме или дифференциальным уравнением первого порядка разрешенным относительно производной.
\end{definition}
\begin{remark}
Множество $\dom f \subset \mathbb R^2$ иногда называют областью (условно) задания уравнения $y' = f(x, y)$. Например, уравнение
$$
y' = \frac{1}{x^2}
$$	
имеет областью задания множество 
$$
\dom f = \left\{(x, y) \in \mathbb R^2: \ x \neq 0\right\}.
$$
\end{remark}
Выясним геометрический смысл уравнения (\ref{de1n}). Пусть $\varphi(x)$ -- решение этого уравнения на $\langle a,  b \rangle$. Тогда
$$
\varphi'(x) \equiv f(x, \varphi(x)), \quad x \in \langle a,  b \rangle.
$$
Пусть $x_0 \in \langle a,  b \rangle$, $y_0 = \varphi(x_0)$, тогда
$$
\varphi'(x_0) = f(x_0, \varphi(x_0)) = f(x_0, y_0)
$$
и оказывается, что значение функции $f$ в точке $(x_0, y_0)$ задает тангенс угла наклона касательной к интегральной кривой $y = \varphi(x)$ в точке $x_0$.
\begin{definition}
Сопоставим каждой точке $\dom f(x, y)$ вектор, направленный под углом $\arctg f(x, y)$ к положительному направлению оси $Ox$. Полученное векторное поле назовем полем направлений. 

Линии уровня функции, то есть множества $f(x, y) = k$, назовем изоклинами.
\end{definition}
Поле направлений позволяет визуально прикинуть поведение интегральных кривых. 
\begin{example}
Рассмотрим дифференциальное уравнение 
$$
x + yy' = 0.
$$	
Пусть $y' = k$. Тогда изоклины имеют следующий вид:
$$
y(x) = -\frac{x}{k}, \quad k \neq 0,
$$
и $x(y) = 0$ при $k = 0$. 

Видно, что на прямой $y = 0$ векторы направлены под углом $\pi/2$, на прямой $y = x$ под углом $-\pi/4$, а на прямой $x = 0$ -- под углом $0$, при этом изменение угла происходит монотонно. Визуально подтверждается, что интегральными кривыми данного уравнения являются окружности.
\end{example}
Анализ поведения изоклин или, что то же самое, поля направлений, может давать и более хитрые результаты.
\begin{example}
Рассмотрим уравнение $y' = y + x$. Изоклинами данного уравнения являются прямые $y = -x + k$.

Заметим, что рассматриваемое уравнение имеет решением функцию $y = -x-1$, что совпадает с одной из изоклин (при $k = -1$). Заметив это, получаем (законность такой операции будет пояснена сильно позже):
$$
y'' = y' + 1 = x + y + 1,
$$
а значит интегральные кривые меняют характер выпуклости в зависимости от положения относительно прямой $y = -x - 1$. 
\end{example}
Построение изоклин и поля направлений намекают на приближенный способ построения решения уравнения (\ref{de1n}) -- на построение ломаных Эйлера. 
\begin{definition}
Ломаные Эйлера -- ломаные, имеющие началом точку $(x_0, y_0)$, концы которых задаются следующими соотношениями:
$$
x_{k} = x_{k - 1} + h, \quad y_{k+1} = y_k + f(x_k, y_k)h.
$$	
\end{definition}
\begin{remark}
Написанные формулы запоминать не надо. Они получаются из приближения производной разностным отношением. Так как при достаточно малых $h$
$$
y'(x_0) \approx \frac{y(x_0 + h) - y(x_0)}{h},
$$	
то
$$
y(x_0 + h) = y(x_0) + f(x_0, y_0)h,
$$
и тем точнее приближение, чем меньше $h$. Вводя соответствующие обозначения и дальше проводя рассуждения для точки $(x_0 + h, y(x_0+ h))$, приходим к озвученным формулам.
\end{remark}
\begin{example}
Надо нарисовать	
\end{example}
\subsection{Уравнение в симметричной форме}
То, что
$$
y' = \frac{dy}{dx}
$$
известно из анализа. Хотя в анализе и учат, что это -- единый символ, а не дробь, Лейбниц вряд ли бы вводил его в обиход, если бы с ним нельзя было обращаться «удобным» образом. Все потому, что $dx$ и $dy$ -- это не какие-то таинственные бесконечно малые, это -- функции от вектора. 

Рассмотрим уравнение (\ref{de1n}) и перепишем его с использованием нашей договоренности:
$$
y' = f(x, y) \ \Leftrightarrow \ dy = f(x,y)dx.
$$
Обобщим это в следующем определении.
\begin{definition}
Уравнение вида
\be\la{de1s}
P(x, y)dx + Q(x, y)dy = 0
\ee
называется дифференциальным уравнением в симметричной форме, или уравнением в дифференциалах.
\end{definition}
В приведенном уравнении переменные входят равноправно.
\begin{remark}
Понятно, что ранее введенное уравнение (\ref{de1n}) -- это частный случай только что введенного уравнения (\ref{de1s}) при 
$$
Q(x, y) \equiv 1, \quad P(x, y) = -f(x, y).
$$
\end{remark}
\begin{remark}
Как и ранее, множество $\dom P(x, y) \cap \dom Q(x,y)$ иногда называют областью (условно) задания уравнения (\ref{de1s}).
\end{remark}

Написанное выражение по своей сути задает дифференциальную форму и не является дифференциальным уравнением в смысле ранее введенного определения. Именно поэтому нам придется отдельно дать определение решения данного уравнения.
\begin{definition}
	Вектор-функция $x = \varphi(t)$, $y = \psi(t)$ называется решением уравнения (\ref{de1s}) на промежутке $\langle \alpha, \beta \rangle$, если
$$
P(\varphi(t), \psi(t))\varphi'(t) + Q(\varphi(t), \psi(t))\psi'(t) \equiv 0, \quad t \in \langle \alpha, \beta \rangle,
$$
и 
$$
|\varphi'(t)| + |\psi'(t)| \neq 0, \quad t \in \langle \alpha, \beta \rangle.
$$
\end{definition}
\begin{remark}
Полезно отметить, что ранее введенное определение решения является частным случаем теперешнего. Полагая 
$$
x = t, \quad y = \psi(t) = \psi(x),
$$
приходим к прежнему определению.

Второе условие говорит о том, что в каждой точке у кривой $(\varphi(t), \psi(t))$ существует касательный вектор.
\end{remark}
\begin{remark}
Как и ранее, интегральной кривой назовем множество $(\varphi(t), \psi(t))$, $t \in \langle \alpha, \beta \rangle$ -- годограф соответствующей вектор-функции.
\end{remark}
\begin{example}
Рассмотрим уравнение $x + yy' = 0$.	В дифференциалах оно переписывается так:
$$
xdx + ydy = 0.
$$
Легко проверить, что вектор-функция $(C\cos t, C \sin t)$, $C > 0$ -- его решение на $\mathbb R$. Перед нами -- параметрические уравнения окружности радиуса $C$.
\end{example}
Попробуем связать вновь введенное уравнение и ранее рассмотренное. Для этого, в частности, предложим следующее определение.
\begin{definition}
Точки, в которых $|P(x, y)| + |Q(x, y)| = 0$, называются особыми точками уравнения (\ref{de1s}).
\end{definition}
\begin{remark}
Простая мотивировка рассмотрения особых и не особых (обыкновенных) точек заключается в следующем. В неособых точках уравнение может быть переписано либо в виде
$$
y' = -\frac{P(x, y)}{Q(x, y)},
$$
либо в виде
$$
x' = -\frac{Q(x, y)}{P(x, y)},
$$
что приводит нас к уравнению типа (\ref{de1n}). Более хитрая мотивировка будет изучена нами позднее, как и понятие особой точки. Если $P, Q$ -- непрерывные функции, то множество особых точек уравнения (\ref{de1s}) замкнуто
\end{remark}
\begin{remark}
Понятно, что как и ранее в каждой неособой точке области задания уравнения (\ref{de1s}) вектор $(P(x, y), Q(x,y))$ задает векторное поле. В особых точках приведенное поле не определяет направлений.

Написанное в определении решения равенство на самом деле говорит об ортогональности данного векторного поля (поля ортогональных направлений) к множеству касательных векторов $(\varphi'(t), \psi'(t))$.
\end{remark}
Покажем, что написанные рассуждения и правда приводят к одним и тем же сущностям.
\begin{definition}
Говорят, что дифференциальные уравнения эквивалентны, если множества их решений совпадают.	
\end{definition}
\begin{lemma}
Уравнения
$$
y' = f(x, y) \quad \text{и} \quad dy = f(x, y)dx
$$	
эквивалентны.
\end{lemma}
\begin{proof}
Области задания данных уравнений, очевидно, одинаковы.

Пусть $y = \varphi(x)$ -- решение первого уравнения на $\langle a, b \rangle$. Пусть $x = t$, $y = \varphi(t)$, $t \in \langle a, b \rangle$, тогда
$$
f(t, \varphi(t)) - \varphi'(t) = f(x, \varphi(x)) - \varphi'(x) \equiv 0.
$$

Наоборот, пусть $x = \varphi(t)$, $y = \psi(t)$ -- решение второго уравнения на $\langle \alpha, \beta \rangle$. Тогда
$$
\psi'(t) \equiv f(\varphi(t), \psi(t))\varphi'(t), \quad t \in \langle \alpha, \beta \rangle.
$$
Заметим, что если $\varphi'(t_0) = 0$, то, необходимо, $\psi'(t_0) = 0$, что противоречит определению решения уравнения (\ref{de1s}). Значит, $\varphi'(t) \neq 0$ на $\langle \alpha, \beta \rangle$, откуда, согласно теореме Дарбу, $\varphi'(t)$ сохраняет знак, а значит $\varphi(t)$ -- монотонная функция, имеющая непрерывную обратную функцию $t = \varphi^{-1}(x)$. По теореме о производной функции, заданной параметрически,
$$
\frac{d}{dx} \psi(\varphi^{-1}(x)) = \frac{\psi'(t)}{\varphi'(t)} = f(\varphi(t), \psi(t)) = f(x, \psi(\varphi^{-1}(x)),
$$
откуда $y = \psi(\varphi^{-1}(x))$ -- решение первого уравнения.
\end{proof}
\subsection{Задача Коши}
Как мы видели ранее, множество решений дифференциального уравнения обычно бесконечно. Конечным оно может быть лишь в весьма вычурных случаях, вроде уравнения
$$
y' = \sqrt{-y^2},
$$
имеющего лишь одно решение $y \equiv 0$ на $\mathbb R$. Интересным оказывается вопрос нахождения интегральной кривой, проходящей через конкретную точку плоскости.
\begin{definition}
Задачей Коши для уравнения 
\be\la{zk1}
y' = f(x, y)
\ee
называют задачу нахождения его частного решения, удовлетворяющего условию
\be\la{zk2}
y(x_0)=y_0.
\ee
\end{definition}един
В дальнейшем, для краткости, мы будем говорить о решении ЗК (\ref{zk1})-(\ref{zk2}).
\begin{example}
Для уравнения 
$$
y' = \frac{1}{x^2}
$$	
уже были построены общие решения 
$$
y = -\frac{1}{x} + C, \quad C \in \mathbb R,
$$
на $(-\infty, 0)$ и $(0, +\infty)$, соответсвенно. Для любой точки $(x_0, y_0)$, у которой $x_0 \neq 0$, существует и при том единственное решение соответствующей ЗК (\ref{zk1})-(\ref{zk2}):
$$
y_0 = -\frac{1}{x_0} + C \ \Rightarrow \ C = y_0 + \frac{1}{x_0},
$$
определенное на $(-\infty, 0)$, если $x_0 < 0$, и на $(0, +\infty)$, если $x_0 > 0$.
\end{example}
Понятно, что после введения такого определения должны возникнуть следующие вопросы: 
\begin{enumerate}
\item Всегда ли существует решение задачи Коши?
\item Если решение существует, то единственно ли оно?
\end{enumerate}
В общем случае на оба вопроса ответ оказывается отрицательным.
\begin{example}
	Рассмотрим уравнение
$$
y' = 3 \sqrt[3]{y^2}.
$$
Несложно проверить, что семейство $y = (x + C)^3$, $C \in \mathbb R$, удовлетворяет написанному уравнению. Это семейство не включает в себя отдельно стоящего решения $y = 0$. Заметим, что через каждую точку $(x_0, 0)$ теперь проходит два решения:
$$
y = 0 \quad \text{и} \quad y = (x - x_0)^3.
$$
Более того, общее решение уравнения дается не только такими функциями, но, скорее, склейками вида
$$
y(x) = \begin{cases}
(x - A)^3, & x \leq A\\
0, & A < x \leq B \\
(x - B)^3, & x > B
 \end{cases}.
$$
При этом можно считать, что $A = -\infty$, $B = +\infty$. В этих случаях «нет» первого и последнего кусков, соответсвенно.

Решение $y \equiv 0$ называется особым, и именно поэтому возникают такие «трудности».
\end{example}
\begin{definition}
Решение $\varphi$ на $\langle a, b \rangle$ уравнения (\ref{de1n}) называется особым, если для любой точки $x_0 \in \langle a, b \rangle$ найдется решение $\psi$ того же уравнения, что
$$
\varphi(x_0) = \psi(x_0),
$$
но для любой окрестности $U(x_0)$ выполняется $\varphi \not\equiv \psi$ в $U(x_0) \cap \langle a, b \rangle$.
\end{definition}
Итого, особое в каждой точке своего множества задания касается какого-то другого решения -- решения, в точности не совпадающего с особым с коль угодно малой окрестности. В примере выше $y \equiv 0$ и есть то самое особое решение.

Приведем (сейчас -- без доказательства) некоторые достаточные условия существования и единственности решения ЗК (\ref{zk1})-(\ref{zk2}).
\begin{theorem}[О существовании решения ЗК]
	Пусть $G$ -- область в $\mathbb R^2$, $f \in C(G)$. Тогда для любой точки $(x_0, y_0) \in G$ существует решение ЗК (\ref{zk1})-(\ref{zk2}).
\end{theorem}
\begin{theorem}[О единственности решения ЗК]
	Пусть $G$ -- область в $\mathbb R^2$, $f, f'_y \in C(G)$, $\varphi_1, \varphi_2$ -- два решения ЗК (\ref{zk1})-(\ref{zk2}) на $ \langle a, b \rangle$ такие, что
	$$
	(x, \varphi_1(x)) \in G, \quad (x, \varphi_2(x)) \in G, \quad x \in \langle a, b \rangle.
	$$
Тогда $\varphi_1 \equiv \varphi_2$ на $\langle a, b \rangle$.
\end{theorem}
Итак, последняя теорема говорит о том, что в случае непрерывности $f, f'_y$ в области $G$ любые два решения одной и той же ЗК совпадают на всем интервале их задания, если только они не покидают $G$. Иными словами, вся область $G$ покрывается непересекающимися кривыми. Понятно, что промежуток задания решения, вообще говоря, зависит от начальных данных $(x_0, y_0)$.
\section{Некоторые уравнения первого порядка, интегрируемые в квадратурах}
В этом пункте мы рассмотрим некоторые алгоритмы, позволяющие решать (и часто даже находить общее решение) некоторых типов дифференциальных уравнений первого порядка. Если все решения уравнения явно или неявно выражаются через простейшие функции при помощи конечного числа арифметических операций, суперпозиций и операций нахождения первообразных, то говорят, что уравнение интегрируется в квадратурах.
\subsection{Уравнение с разделенными переменными}
Начнем, наверное, самого простого уравнения -- уравнения с разделенными переменными.
\begin{definition}
Уравнение 
\be\la{razdper}
P(x)dx + Q(y)dy = 0
\ee
называется уравнением с разделенными переменными.
\end{definition}
Название данного уравнения обусловлено тем, что при $dx$ и $dy$ стоят только функции, зависящие от $x$ и $y$, соответсвенно.
\begin{remark}
Понятно, что так как переменные разделены, то хочется записать общее решение уравнения (\ref{razdper}) в виде 
$$
\int P(x) \, dx + \int Q(y) \, dy = C.
$$	
Впрочем, это обстоятельство требует как пояснения, так и уточнения.
\end{remark}
\begin{theorem}
Пусть $P \in C \langle a, b \rangle$, $Q \in C \langle c, d \rangle$. Вектор-функция $x=\varphi(t)$, $y = \psi(t)$ является решением уравнения (\ref{razdper}) на $\langle \alpha, \beta \rangle$ тогда и только тогда, когда $\varphi, \psi$ дифференцируемы на $\langle \alpha, \beta \rangle$ и при некотором $C$ удовлетворяют уравнению
\be\la{razdpersol}
\int P(x) \, dx + \int Q(y) \, dy = C.
\ee	
\end{theorem}
\begin{proof}
Докажем необходимость. Дифференцируемость $\varphi, \psi$ следует из определения решения уравнения (\ref{de1s}). Покажем, что они удовлетворяют уравнению (\ref{razdpersol}) при некотором $C$. Пусть
$$
A(x) = \int P(x) \, dx, \quad B(x) = \int Q(x) \, dx,
$$
-- какие-либо первообразные функций $P$ и $Q$, соответсвенно. Так как
$$
P(\varphi(t))\varphi'(t) + Q(\psi(t))\psi'(t) \equiv 0, \quad t \in \langle \alpha, \beta \rangle,
$$
то, интегрируя написанное равенство, получаем
$$
A(\varphi(t)) + B(\psi(t)) \equiv C, \quad t \in \langle \alpha, \beta \rangle.
$$
Достаточность. Продифференцируем предыдущее равенство по переменной $t$ на $\langle \alpha, \beta \rangle$, получим
$$
P(\varphi(t))\varphi'(t) + Q(\psi(t))\psi'(t) \equiv 0, \quad t \in \langle \alpha, \beta \rangle,
$$
что и доказывает утверждение.
\end{proof}
Вернемся к уже рассмотренному ранее примеру.
\begin{example}
Найти общий интеграл уравнения
$$
xdx + ydy = 0.
$$
Пользуясь формулой (\ref{razdpersol}), получаем:
$$
x^2 + y^2 = 2C.
$$
Понятно, что это уравнение определяет решения лишь при $C > 0$.
\end{example}
\subsection{Уравнение с разделяющимися переменными}
\begin{definition}
Уравнение 
\be\la{razdper2}
P_1(x)Q_1(y)dx + P_2(x)Q_2(y)dy = 0
\ee
называется уравнением с разделяющимися переменными.
\end{definition}
Название этого уравнения обусловлено тем, что путем деления на $Q_1(y)P_2(x)$ уравнение (\ref{razdper2}) приводится к уравнению (\ref{razdper}) вида:
$$
\frac{P_1(x)}{P_2(x)} dx + \frac{Q_2(y)}{Q_1(y)}dy = 0.
$$
В то же время, осуществляя соответствующее деление нужно удостовериться, что мы не теряем решений. Пусть $P_1, P_2 \in C \langle a, b \rangle$, $Q_1, Q_2 \in C \langle c, d \rangle$. 

Если $Q_1(y_0)$ = 0, то $y \equiv y_0$, $x \in \langle a, b \rangle$ -- решение. Для поиска других интегральных кривых множество $\langle a, b \rangle \times \langle c, d \rangle$ разбивается на две части с общей границей $y = y_0$.

Аналогично, если $P_2(x_0)$ = 0, то $x \equiv x_0$, $y \in \langle c, d \rangle$ -- решение. Для поиска других интегральных кривых множество $\langle a, b \rangle \times \langle c, d \rangle$ разбивается на две части с общей границей $x = x_0$.

В остальном решение уравнения (\ref{razdper2}) сводится к формуле (\ref{razdpersol}) и соответствующей теореме.
\begin{example}
Решить уравнение 
$$
xdy - 2ydx = 0.
$$	
Область задания данного уравнения -- это $\mathbb R^2$. Заметим, что $x = 0$ и $y = 0$ -- решения на $\mathbb R$. При делении на $xy$ приходим к уравнению
$$
\frac{dy}{y} - 2\frac{dx}{x} = 0.
$$
Будем решать его сначала в первой четверти то есть при $x, y > 0$. Согласно формуле (\ref{razdpersol}), общее решение такого уравнения задается соотношением
$$
\ln |y| - 2 \ln|x| = C \ \Leftrightarrow \ \ln |y| = \ln(Ax^2) \ \Leftrightarrow \ |y| = Ax^2, \quad A = e^C.
$$
откуда, в рассматриваемой первой четверти,
$$
y = Ax^2, \quad A > 0.
$$
Такое же решение будет и во второй четверти. В третьей и четвертых четвертях получим решения
$$
y = Ax^2, \quad A < 0.
$$
Интегральные кривые в «диагональных» четвертях допускают гладкое сшивание как друг с другом, так и с прямой $y = 0$. В итоге, общее решение задается следующими кусочными функциями:
$$
y(x) = \begin{cases}
Ax^2, & x < 0 \\	
Bx^2, & x \geq 0
 \end{cases},
$$
где $A, B$ -- любые, а также $x(y) \equiv 0$ на $\mathbb R$.
\end{example}
Склейки в решениях бывают не всегда.
\begin{example}
Решить уравнение 
$$
xdy - ydx = 0.
$$	
Область задания данного уравнения -- это $\mathbb R^2$. Заметим, что $x = 0$ и $y = 0$ -- решения на $\mathbb R$. При делении на $xy$ приходим к уравнению
$$
\frac{dy}{y} - \frac{dx}{x} = 0.
$$
Будем решать его сначала в первой четверти то есть при $x, y > 0$. Согласно формуле (\ref{razdpersol}), общее решение такого уравнения задается соотношением
$$
\ln |y| - \ln |x| = C \ \Leftrightarrow \ |y| = A|x|, \quad A = e^C.
$$
В первой и третьей четвертях имеем $y = Ax$, $A > 0$, а во второй и четвертой -- $y = Ax$, $A < 0$. Понятно, что с требованием дифференцируемости в начале координат можно объединить лишь части одной и той же кривой. Тем самым, общее решение состоит из прямых $y = Ax$ и прямой $x(y) \equiv 0$. Составных решений данное уравнение не имеет.
\end{example}
\subsection{Однородное уравнение}
Введем понятие однородной функции.
\begin{definition}
Функция $F(x, y)$ называется однородной степени однородности $\alpha$, если для всех допустимых $x, y, \lambda$ выполняется
$$
F(\lambda x, \lambda y) = \lambda^\alpha F(x, y).
$$	
\end{definition}
\begin{definition}
Уравнение 
\be\la{odnorod}
P(x, y)dx + Q(x, y)dy = 0,
\ee
где $P$ и $Q$ -- однородные функции одинаковой степени однородности, называется однородным уравнением.	
\end{definition}
\begin{remark}
Однородное уравнение заменой $y = z(x)x$ приводится к уравнению с разделяющимися переменными. Действительно, действуя формально, так как $dy = zdx + xdz$, получаем
$$
P(x, zx)dx + Q(x, zx)(zdx + xdz) = 0.
$$	
Учитывая, что $P$ и $Q$ -- однородные функции степени однородности $\alpha$, получаем
$$
x^\alpha(P(1, z) + zQ(1,z))dz + x^{\alpha +1}Q(1, z)dz = 0б
$$
что является уравнением с разделяющимися переменными (\ref{razdper2}).
\end{remark}
\begin{example}
	
\end{example}












\end{document} 